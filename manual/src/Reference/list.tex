\hypertarget{list}{%
\section{List}\label{list}}

Lists are collection objects that contain a sequence of values each
associated with an integer index.

Create a list like this:

\begin{lstlisting}
var list = [1, 2, 3]
\end{lstlisting}

Look up values using index notation:

\begin{lstlisting}
list[0]
\end{lstlisting}

Indexing can also be done with slices: list{[}0..2{]}
list{[}{[}0,1,3{]}{]}

You can change list entries like this:

\begin{lstlisting}
list[0] = "Hello"
\end{lstlisting}

Create an empty list:

\begin{lstlisting}
var list = []
\end{lstlisting}

Loop over elements of a list:

\begin{lstlisting}
for (i in list) print i
\end{lstlisting}

\hypertarget{append}{%
\subsection{Append}\label{append}}

Adds an element to the end of a list:

\begin{lstlisting}
var list = []
list.append("Foo")
\end{lstlisting}

\hypertarget{insert}{%
\subsection{Insert}\label{insert}}

Inserts an element into a list at a specified index:

\begin{lstlisting}
var list = [1,2,3]
list.insert(1, "Foo")
print list // prints [ 1, Foo, 2, 3 ]
\end{lstlisting}

\hypertarget{pop}{%
\subsection{Pop}\label{pop}}

Remove the last element from a list, returning the element removed:

\begin{lstlisting}
print list.pop()
\end{lstlisting}

If an integer argument is supplied, returns and removes that element:

\begin{lstlisting}
var a = [1,2,3]
print a.pop(1) // prints '2'
print a        // prints [ 1, 3 ]
\end{lstlisting}

\hypertarget{sort}{%
\subsection{Sort}\label{sort}}

Sorts the contents of a list into ascending order:

\begin{lstlisting}
list.sort()
\end{lstlisting}

Note that this sorts the list ``in place'' (i.e.~it modifies the order
of the list on which it is invoked) and hence returns \texttt{nil}.

You can provide your own function to use to compare values in the list

\begin{lstlisting}
list.sort(fn (a, b) a-b)
\end{lstlisting}

This function should return a negative value if \texttt{a\textless{}b},
a positive value if \texttt{a\textgreater{}b} and \texttt{0} if
\texttt{a} and \texttt{b} are equal.

\hypertarget{order}{%
\subsection{Order}\label{order}}

Returns a list of indices that would, if used in order, would sort a
list. For example

\begin{lstlisting}
var list = [2,3,1]
print list.order() // expect: [2,0,1]
\end{lstlisting}

would produce \texttt{{[}2,0,1{]}}

\hypertarget{remove}{%
\subsection{Remove}\label{remove}}

Remove any occurrences of a value from a list:

\begin{lstlisting}
var list = [1,2,3]
list.remove(1)
\end{lstlisting}

\hypertarget{ismember}{%
\subsection{ismember}\label{ismember}}

Tests if a value is a member of a list:

\begin{lstlisting}
var list = [1,2,3]
print list.ismember(1) // expect: true
\end{lstlisting}

\hypertarget{add}{%
\subsection{Add}\label{add}}

Join two lists together:

\begin{lstlisting}
var l1 = [1,2,3], l2 = [4, 5, 6]
print l1+l2 // expect: [1,2,3,4,5,6]
\end{lstlisting}

\hypertarget{tuples}{%
\subsection{Tuples}\label{tuples}}

Generate all possible 2-tuples from a list:

\begin{lstlisting}
var t = [ 1, 2, 3].tuples(2)
\end{lstlisting}

produces
\texttt{{[}\ {[}\ 1,\ 1\ {]},\ {[}\ 1,\ 2\ {]},\ {[}\ 1,\ 3\ {]}\ ...\ {]}}.

\hypertarget{sets}{%
\subsection{Sets}\label{sets}}

Generate all possible sets of order 2 from a list.

\begin{lstlisting}
var t = [ 1, 2, 3 ].sets(2)
\end{lstlisting}

produces
\texttt{{[}\ {[}\ 1,\ 2\ {]},\ {[}\ 1,\ 3\ {]},\ {[}\ 2,\ 3\ {]}\ {]}}.

Note that sets include only distinct elements from the list (no element
is repeated) and ordering is unimportant, hence only one of
\texttt{{[}\ 1,\ 2\ {]}} and \texttt{{[}\ 2,\ 1\ {]}} is returned.
