\hypertarget{povray}{%
\section{POVRay}\label{povray}}

The \texttt{povray} module provides integration with POVRay, a popular
open source ray-tracing package for high quality graphical rendering. To
use the module, first import it:

\begin{lstlisting}
import povray
\end{lstlisting}

To raytrace a graphic, begin by creating a \texttt{POVRaytracer} object:

\begin{lstlisting}
var pov = POVRaytracer(graphic)
\end{lstlisting}

Create a .pov file that can be run with POVRay:

\begin{lstlisting}
pov.write("out.pov")
\end{lstlisting}

Create, render and display a scene using POVRay:

\begin{lstlisting}
pov.render("out.pov")
\end{lstlisting}

This also creates the .png file for the scene.

The \texttt{POVRaytracer} constructor supports a number of optional
arguments:

\begin{itemize}

\item
  \texttt{antialias} - whether to antialias the output or not
\item
  \texttt{width} - image width
\item
  \texttt{height} - image height
\item
  \texttt{viewangle} - camera angle (higher means wider view)
\item
  \texttt{viewpoint} - position of camera
\end{itemize}

The \texttt{render} method supports two optional boolean arguments:

\begin{itemize}

\item
  \texttt{quiet} - whether to suppress the parser and render statistics
  from \texttt{povray} or not (\texttt{false} by default)
\item
  \texttt{display} - whether to turn on the graphic display while
  rendering or not (\texttt{true} by default)
\end{itemize}
