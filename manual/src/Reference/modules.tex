\hypertarget{modules}{%
\section{Modules}\label{modules}}

Morpho is extensible and provides a convenient module system that works
like standard libraries in other languages. Modules may define useful
variables, functions and classes, and can be made available using the
\texttt{import} keyword. For example,

\begin{lstlisting}
import color
\end{lstlisting}

loads the \texttt{color} module that provides functionality related to
color.

You can create your own modules; they're just regular morpho files that
are stored in a standard place. On UNIX platforms, this is
\texttt{/usr/local/share/morpho/modules}.

\hypertarget{import}{%
\subsection{Import}\label{import}}

Import provides access to the module system and including code from
multiple source files.

To import code from another file, use import with the filename:

\begin{lstlisting}
import "file.morpho"
\end{lstlisting}

which immediately includes all the contents of \texttt{"file.morpho"}.
Any classes, functions or variables defined in that file can now be
used, which allows you to divide your program into multiple source
files.

Morpho provides a number of built in modules--and you can write your
own--which can be loaded like this:

\begin{lstlisting}
import color
\end{lstlisting}

which imports the \texttt{color} module.

You can selectively import symbols from a modules by using the
\texttt{for} keyword:

\begin{lstlisting}
import color for HueMap, Red
\end{lstlisting}

which imports only the \texttt{HueMap} class and the \texttt{Red}
variable.
