\hypertarget{complex}{%
\section{Complex}\label{complex}}

Morpho provides complex numbers. The keyword \texttt{im} is used to
denote the imaginary part of a complex number:

\begin{lstlisting}
var a=1+5im 
print a*a
\end{lstlisting}

Print values on the unit circle in the complex plane:

\begin{lstlisting}
import constants 
for (phi in 0..Pi:Pi/5) print exp(im*phi)
\end{lstlisting}

Get the real and imaginary parts of a complex number:

\begin{lstlisting}
print real(a) 
print imag(a) 
\end{lstlisting}

or alternatively:

\begin{lstlisting}
print a.real()
print a.imag() 
\end{lstlisting}

\hypertarget{angle}{%
\section{Angle}\label{angle}}

Returns the angle \texttt{phi} associated with the polar representation
of a complex number \texttt{r*exp(im*phi)}:

\begin{lstlisting}
print z.angle() 
\end{lstlisting}

\hypertarget{conj}{%
\section{Conj}\label{conj}}

Returns the complex conjugate of a number:

\begin{lstlisting}
print z.conj() 
\end{lstlisting}
