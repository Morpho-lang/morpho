\hypertarget{delaunay}{%
\section{Delaunay}\label{delaunay}}

The \texttt{delaunay} module creates Delaunay triangulations from point
clouds. It is dimensionally independent, so generates tetrahedra in 3D
and higher order simplices beyond.

To use the module, first import it:

\begin{lstlisting}
import delaunay
\end{lstlisting}

To create a Delaunary triangulation from a list of points:

\begin{lstlisting}
var pts = []
for (i in 0...100) pts.append(Matrix([random(), random()]))
var del=Delaunay(pts)
print del.triangulate()
\end{lstlisting}

The module also provides \texttt{DelaunayMesh} to directly create meshes
from Delaunay triangulations.

\hypertarget{triangulate}{%
\section{Triangulate}\label{triangulate}}

The \texttt{triangulate} method performs the delaunay triangulation. To
use it, first construct a \texttt{Delaunay} object with the point cloud
of interest:

\begin{lstlisting}
var del=Delaunay(pts)
\end{lstlisting}

Then call \texttt{triangulate}:

\begin{lstlisting}
var tri = del.triangulate()
\end{lstlisting}

This returns a list of triangles
\texttt{{[}\ {[}i,\ j,\ k{]},\ ...\ {]}}.

\hypertarget{delaunaymesh}{%
\section{DelaunayMesh}\label{delaunaymesh}}

The \texttt{DelaunayMesh} constructor function creates a \texttt{Mesh}
object directly from a point cloud using the Delaunay triangulator.

\begin{lstlisting}
var pts = []
for (i in 0...100) pts.append(Matrix([random(), random()]))
var m=DelaunayMesh(pts)
Show(plotmesh(m))
\end{lstlisting}

You can control the output dimension of the mesh (e.g.~to create a 2D
mesh embedded in 3D space) using the optional \texttt{outputdim}
property.

\begin{lstlisting}
var m = DelaunayMesh(pts, outputdim=3)
\end{lstlisting}

\hypertarget{circumsphere}{%
\section{Circumsphere}\label{circumsphere}}

The \texttt{Circumsphere} class calculates the circumsphere of a set of
points, i.e.~a sphere such that all the points are on the surface of the
sphere. It is used internally by the \texttt{delaunay} module.

Create a \texttt{Circumsphere} from a list of points and a triangle
specified by indices into that list:

\begin{lstlisting}
var sph = Circumsphere(pts, [i,j,k]) 
\end{lstlisting}

Test if an arbitrary point is inside the \texttt{Circumsphere} or not:

\begin{lstlisting}
print sph.pointinsphere(pt)
\end{lstlisting}
