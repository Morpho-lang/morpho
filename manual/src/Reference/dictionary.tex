\hypertarget{dictionary}{%
\section{Dictionary}\label{dictionary}}

Dictionaries are collection objects that associate a unique \emph{key}
with a particular \emph{value}. Keys can be any kind of morpho value,
including numbers, strings and objects.

An example dictionary mapping states to capitals:

\begin{lstlisting}
var dict = { "Massachusetts" : "Boston",
             "New York" : "Albany",
             "Vermont" : "Montpelier" }
\end{lstlisting}

Look up values by a given key with index notation:

\begin{lstlisting}
print dict["Vermont"]
\end{lstlisting}

You can change the value associated with a key, or add new elements to
the dictionary like this:

\begin{lstlisting}
dict["Maine"]="Augusta"
\end{lstlisting}

Create an empty dictionary using the \texttt{Dictionary} constructor
function:

\begin{lstlisting}
var d = Dictionary()
\end{lstlisting}

Loop over keys in a dictionary:

\begin{lstlisting}
for (k in dict) print k
\end{lstlisting}

The \texttt{keys} method returns a Morpho List of the keys.

\begin{lstlisting}
var keys = dict.keys() // will return ["Massachusetts", "New York", "Vermont"]
\end{lstlisting}

The \texttt{contains} method returns a Bool value for whether the
Dictionary contains a given key.

\begin{lstlisting}
print dict.contains("Vermont") // true
print dict.contains("New Hampshire") // false
\end{lstlisting}

The \texttt{remove} method removes a given key from the Dictionary.

\begin{lstlisting}
dict.remove("Vermont")
print dict // { New York : Albany, Massachusetts : Boston }
\end{lstlisting}

The \texttt{clear} method removes all the (key, value) pairs fromt the
dictionary, resulting in an empty dictionary.

\begin{lstlisting}
dict.clear()

print dict // {  }
\end{lstlisting}
