\hypertarget{functionals}{%
\section{Functionals}\label{functionals}}

A number of \texttt{functionals} are available in Morpho. Each of these
represents an integral over some \texttt{Mesh} and \texttt{Field}
objects (on a particular \texttt{Selection}) and are used to define
energies and constraints in an \texttt{OptimizationProblem} provided by
the \texttt{optimize} module.

Many functionals are built in. Additional functionals are available by
importing the \texttt{functionals} module:

\begin{lstlisting}
import functionals
\end{lstlisting}

Functionals provide a number of standard methods:

\begin{itemize}

\item
  \texttt{total}(mesh) - returns the value of the integral with a
  provided mesh, selection and fields
\item
  \texttt{integrand}(mesh) - returns the contribution to the integral
  from each element
\item
  \texttt{gradient}(mesh) - returns the gradient of the functional with
  respect to vertex motions.
\item
  \texttt{fieldgradient}(mesh, field) - returns the gradient of the
  functional with respect to components of the field
\end{itemize}

Each of these may be called with a mesh, a field and a selection.

\hypertarget{length}{%
\subsection{Length}\label{length}}

A \texttt{Length} functional calculates the length of a line element in
a mesh.

Evaluate the length of a circular loop:

\begin{lstlisting}
import constants
import meshtools
var m = LineMesh(fn (t) [cos(t), sin(t), 0], 0...2*Pi:Pi/20, closed=true)
var le = Length()
print le.total(m)
\end{lstlisting}

\hypertarget{areaenclosed}{%
\subsection{AreaEnclosed}\label{areaenclosed}}

An \texttt{AreaEnclosed} functional calculates the area enclosed by a
loop of line elements.

\begin{lstlisting}
var la = AreaEnclosed()
\end{lstlisting}

\hypertarget{area}{%
\subsection{Area}\label{area}}

An \texttt{Area} functional calculates the area of the area elements in
a mesh:

\begin{lstlisting}
var la = Area()
print la.total(mesh)
\end{lstlisting}

\hypertarget{volumeenclosed}{%
\subsection{VolumeEnclosed}\label{volumeenclosed}}

A \texttt{VolumeEnclosed} functional is used to calculate the volume
enclosed by a surface. Note that this estimate may become inaccurate for
highly deformed surfaces.

\begin{lstlisting}
var lv = VolumeEnclosed()
\end{lstlisting}

\hypertarget{volume}{%
\subsection{Volume}\label{volume}}

A \texttt{Volume} functional calculates the volume of volume elements.

\begin{lstlisting}
var lv = Volume()
\end{lstlisting}

\hypertarget{scalarpotential}{%
\subsection{ScalarPotential}\label{scalarpotential}}

The \texttt{ScalarPotential} functional is applied to point elements.

\begin{lstlisting}
var ls = ScalarPotential(potential)
\end{lstlisting}

You must supply a function (which may be anonymous) that returns the
potential. You may optionally provide a function that returns the
gradient as well at initialization:

\begin{lstlisting}
var ls = ScalarPotential(potential, gradient)
\end{lstlisting}

This functional is often used to constrain the mesh to the level set of
a function. For example, to confine a set of points to a sphere:

\begin{lstlisting}
import optimize
fn sphere(x,y,z) { return x^2+y^2+z^2-1 }
fn grad(x,y,z) { return Matrix([2*x, 2*y, 2*z]) }
var lsph = ScalarPotential(sphere, grad)
problem.addlocalconstraint(lsph)
\end{lstlisting}

See the thomson example for use of this technique.

\hypertarget{linearelasticity}{%
\subsection{LinearElasticity}\label{linearelasticity}}

The \texttt{LinearElasticity} functional measures the linear elastic
energy away from a reference state.

You must initialize with a reference mesh:

\begin{lstlisting}
var le = LinearElasticity(mref)
\end{lstlisting}

Manually set the poisson's ratio and grade to operate on:

\begin{lstlisting}
le.poissonratio = 0.2
le.grade = 2
\end{lstlisting}

\hypertarget{equielement}{%
\subsection{EquiElement}\label{equielement}}

The \texttt{EquiElement} functional measures the discrepency between the
size of elements adjacent to each vertex. It can be used to equalize
elements for regularization purposes.

\hypertarget{linecurvaturesq}{%
\subsection{LineCurvatureSq}\label{linecurvaturesq}}

The \texttt{LineCurvatureSq} functional measures the integrated
curvature squared of a sequence of line elements.

\hypertarget{linetorsionsq}{%
\subsection{LineTorsionSq}\label{linetorsionsq}}

The \texttt{LineTorsionSq} functional measures the integrated torsion
squared of a sequence of line elements.

\hypertarget{meancurvaturesq}{%
\subsection{MeanCurvatureSq}\label{meancurvaturesq}}

The \texttt{MeanCurvatureSq} functional computes the integrated mean
curvature over a surface.

\hypertarget{gausscurvature}{%
\subsection{GaussCurvature}\label{gausscurvature}}

The \texttt{GaussCurvature} computes the integrated gaussian curvature
over a surface.

Note that for surfaces with a boundary, the integrand is correct only
for the interior points. To compute the geodesic curvature of the
boundary in that case, you can set the optional flag \texttt{geodesic}
to \texttt{true} and compute the total on the boundary selection. Here
is an example for a 2D disk mesh.

\begin{lstlisting}
var mesh = Mesh("disk.mesh")
mesh.addgrade(1)

var whole = Selection(mesh, fn(x,y,z) true)
var bnd = Selection(mesh, boundary=true)
var interior = whole.difference(bnd)

var gauss = GaussCurvature()
print gauss.total(mesh, selection=interior) // expect: 0
gauss.geodesic = true
print gauss.total(mesh, selection=bnd) // expect: 2*Pi
\end{lstlisting}

\hypertarget{gradsq}{%
\subsection{GradSq}\label{gradsq}}

The \texttt{GradSq} functional measures the integral of the gradient
squared of a field. The field can be a scalar, vector or matrix
function.

Initialize with the required field:

\begin{lstlisting}
var le=GradSq(phi)
\end{lstlisting}

\hypertarget{nematic}{%
\subsection{Nematic}\label{nematic}}

The \texttt{Nematic} functional measures the elastic energy of a nematic
liquid crystal.

\begin{lstlisting}
var lf=Nematic(nn)
\end{lstlisting}

There are a number of optional parameters that can be used to set the
splay, twist and bend constants:

\begin{lstlisting}
var lf=Nematic(nn, ksplay=1, ktwist=0.5, kbend=1.5, pitch=0.1)
\end{lstlisting}

These are stored as properties of the object and can be retrieved as
follows:

\begin{lstlisting}
print lf.ksplay
\end{lstlisting}

\hypertarget{nematicelectric}{%
\subsection{NematicElectric}\label{nematicelectric}}

The \texttt{NematicElectric} functional measures the integral of a
nematic and electric coupling term integral((n.E)\^{}2) where the
electric field E may be computed from a scalar potential or supplied as
a vector.

Initialize with a director field \texttt{nn} and a scalar potential
\texttt{phi}: var lne = NematicElectric(nn, phi)

\hypertarget{normsq}{%
\subsection{NormSq}\label{normsq}}

The \texttt{NormSq} functional measures the elementwise L2 norm squared
of a field.

\hypertarget{lineintegral}{%
\subsection{LineIntegral}\label{lineintegral}}

The \texttt{LineIntegral} functional computes the line integral of a
function. You supply an integrand function that takes a position matrix
as an argument.

To compute \texttt{integral(x\^{}2+y\^{}2)} over a line element:

\begin{lstlisting}
var la=LineIntegral(fn (x) x[0]^2+x[1]^2)
\end{lstlisting}

The function \texttt{tangent()} returns a unit vector tangent to the
current element:

\begin{lstlisting}
var la=LineIntegral(fn (x) x.inner(tangent()))
\end{lstlisting}

You can also integrate functions that involve fields:

\begin{lstlisting}
var la=LineIntegral(fn (x, n) n.inner(tangent()), n)
\end{lstlisting}

where \texttt{n} is a vector field. The local interpolated value of this
field is passed to your integrand function. More than one field can be
used; they are passed as arguments to the integrand function in the
order you supply them to \texttt{LineIntegral}.

\hypertarget{areaintegral}{%
\subsection{AreaIntegral}\label{areaintegral}}

The \texttt{AreaIntegral} functional computes the area integral of a
function. You supply an integrand function that takes a position matrix
as an argument.

To compute integral(x*y) over an area element:

\begin{lstlisting}
var la=AreaIntegral(fn (x) x[0]*x[1])
\end{lstlisting}

You can also integrate functions that involve fields:

\begin{lstlisting}
var la=AreaIntegral(fn (x, phi) phi^2, phi)
\end{lstlisting}

More than one field can be used; they are passed as arguments to the
integrand function in the order you supply them to
\texttt{AreaIntegral}.

\hypertarget{volumeintegral}{%
\subsection{VolumeIntegral}\label{volumeintegral}}

The \texttt{VolumeIntegral} functional computes the volume integral of a
function. You supply an integrand function that takes a position matrix
as an argument.

To compute integral(x\emph{y}z) over an volume element:

\begin{lstlisting}
var la=VolumeIntegral(fn (x) x[0]*x[1]*x[2])
\end{lstlisting}

You can also integrate functions that involve fields:

\begin{lstlisting}
var la=VolumeIntegral(fn (x, phi) phi^2, phi)
\end{lstlisting}

More than one field can be used; they are passed as arguments to the
integrand function in the order you supply them to
\texttt{VolumeIntegral}.

\hypertarget{hydrogel}{%
\subsection{Hydrogel}\label{hydrogel}}

The \texttt{Hydrogel} functional computes the Flory-Rehner energy over
an element:

\begin{lstlisting}
(a*phi*log(phi) + b*(1-phi)+log(1-phi) + c*phi*(1-phi))*V + 
d*(log(phiref/phi)/3 - (phiref/phi)^(2/3) + 1)*V0
\end{lstlisting}

The first three terms come from the Flory-Huggins mixing energy, whereas
the fourth term proportional to d comes from the Flory-Rehner elastic
energy.

The value of phi is calculated from a reference mesh that you provide on
initializing the Functional:

\begin{lstlisting}
var lfh = Hydrogel(mref)
\end{lstlisting}

Here, a, b, c, d and phiref are parameters you can supply (they are
\texttt{nil} by default), V is the current volume and V0 is the
reference volume of a given element. You also need to supply the initial
value of phi, labeled as phi0, which is assumed to be the same for all
the elements. Manually set the coefficients and grade to operate on:

\begin{lstlisting}
lfh.a = 1; lfh.b = 1; lfh.c = 1; lfh.d = 1;
lfh.grade = 2, lfh.phi0 = 0.5, lfh.phiref = 0.1
\end{lstlisting}
