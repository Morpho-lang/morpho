\hypertarget{matrix}{%
\section{Matrix}\label{matrix}}

The Matrix class provides support for matrices. A matrix can be
initialized with a given size,

\begin{lstlisting}
var a = Matrix(nrows,ncols)
\end{lstlisting}

where all elements are initially set to zero. Alternatively, a matrix
can be created from an array,

\begin{lstlisting}
var a = Matrix([[1,2], [3,4]])
\end{lstlisting}

or a Sparse matrix,

\begin{lstlisting}
var a = Sparse([[0,0,1],[1,1,1],[2,2,1]])
var b = Matrix(a)
\end{lstlisting}

You can create a column vector like this,

\begin{lstlisting}
var v = Matrix([1,2])
\end{lstlisting}

Finally, you can create a Matrix by assembling other matrices like this,

\begin{lstlisting}
var a = Matrix([[0,1],[1,0]])
var b = Matrix([[a,0],[0,a]]) // produces a 4x4 matrix 
\end{lstlisting}

Once a matrix is created, you can use all the regular arithmetic
operators with matrix operands, e.g.

\begin{lstlisting}
a+b
a*b
\end{lstlisting}

The division operator is used to solve a linear system, e.g.

\begin{lstlisting}
var a = Matrix([[1,2],[3,4]])
var b = Matrix([1,2])

print b/a
\end{lstlisting}

yields the solution to the system a*x = b.

\hypertarget{assign}{%
\subsection{Assign}\label{assign}}

Copies the contents of matrix B into matrix A:

\begin{lstlisting}
A.assign(B)
\end{lstlisting}

The two matrices must have the same dimensions.

\hypertarget{dimensions}{%
\subsection{Dimensions}\label{dimensions}}

Returns the dimensions of a matrix:

\begin{lstlisting}
var A = Matrix([1,2,3]) // Create a column matrix 
print A.dimensions()    // Expect: [ 3, 1 ]
\end{lstlisting}

\hypertarget{eigenvalues}{%
\subsection{Eigenvalues}\label{eigenvalues}}

Returns a list of eigenvalues of a Matrix:

\begin{lstlisting}
var A = Matrix([[0,1],[1,0]])
print A.eigenvalues() // Expect: [1,-1]
\end{lstlisting}

\hypertarget{eigensystem}{%
\subsection{Eigensystem}\label{eigensystem}}

Returns the eigenvalues and eigenvectors of a Matrix:

\begin{lstlisting}
var A = Matrix([[0,1],[1,0]])
print A.eigensystem() 
\end{lstlisting}

Eigensystem returns a two element list: The first element is a List of
eigenvalues. The second element is a Matrix containing the corresponding
eigenvectors as its columns:

\begin{lstlisting}
print A.eigensystem()[0]
// [ 1, -1 ]
print A.eigensystem()[1]
// [ 0.707107 -0.707107 ]
// [ 0.707107 0.707107 ]
\end{lstlisting}

\hypertarget{inner}{%
\subsection{Inner}\label{inner}}

Computes the Frobenius inner product between two matrices:

\begin{lstlisting}
var prod = A.inner(B)
\end{lstlisting}

\hypertarget{outer}{%
\subsection{Outer}\label{outer}}

Computes the outer produce between two vectors:

\begin{lstlisting}
var prod = A.outer(B)
\end{lstlisting}

Note that \texttt{outer} always treats both vectors as column vectors.

\hypertarget{inverse}{%
\subsection{Inverse}\label{inverse}}

Returns the inverse of a matrix if it is invertible. Raises a
\texttt{MtrxSnglr} error if the matrix is singular. E.g.

\begin{lstlisting}
var m = Matrix([[1,2],[3,4]])
var mi = m.inverse()
\end{lstlisting}

yields the inverse of the matrix \texttt{m}, such that mi*m is the
identity matrix.

\hypertarget{norm}{%
\subsection{Norm}\label{norm}}

Returns a matrix norm. By default the L2 norm is returned:

\begin{lstlisting}
var a = Matrix([1,2,3,4])
print a.norm() // Expect: sqrt(30) = 5.47723...
\end{lstlisting}

You can select a different norm by supplying an argument:

\begin{lstlisting}
import constants
print a.norm(1) // Expect: 10 (L1 norm is sum of absolute values) 
print a.norm(3) // Expect: 4.64159 (An unusual choice of norm)
print a.norm(Inf) // Expect: 4 (Inf-norm corresponds to maximum absolute value)
\end{lstlisting}

\hypertarget{reshape}{%
\subsection{Reshape}\label{reshape}}

Changes the dimensions of a matrix such that the total number of
elements remains constant:

\begin{lstlisting}
var A = Matrix([[1,3],[2,4]])
A.reshape(1,4) // 1 row, 4 columns
print A // Expect: [ 1, 2, 3, 4 ]
\end{lstlisting}

Note that elements are stored in column major-order.

\hypertarget{sum}{%
\subsection{Sum}\label{sum}}

Returns the sum of all entries in a matrix:

\begin{lstlisting}
var sum = A.sum() 
\end{lstlisting}

\hypertarget{transpose}{%
\subsection{Transpose}\label{transpose}}

Returns the transpose of a matrix:

\begin{lstlisting}
var At = A.transpose()
\end{lstlisting}

\hypertarget{trace}{%
\subsection{Trace}\label{trace}}

Computes the trace (the sum of the diagonal elements) of a square
matrix:

\begin{lstlisting}
var tr = A.trace()
\end{lstlisting}

\hypertarget{roll}{%
\subsection{Roll}\label{roll}}

Rotates values in a Matrix about a given axis by a given shift:

\begin{lstlisting}
var r = A.roll(shift, axis)
\end{lstlisting}

Elements that roll beyond the last position are re-introduced at the
first.

\hypertarget{identitymatrix}{%
\subsection{IdentityMatrix}\label{identitymatrix}}

Constructs an identity matrix of a specified size:

\begin{lstlisting}
var a = IdentityMatrix(size)
\end{lstlisting}
