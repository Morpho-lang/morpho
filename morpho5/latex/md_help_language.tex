\mbox{[}comment\mbox{]}\+: \# Morpho language help file \mbox{[}version\mbox{]}\+: \# 0.\+5

\#\+Functions \mbox{[}tag\mbox{]}\+: \# fn \mbox{[}tag\mbox{]}\+: \# func

A function in Morpho is defined with the {\ttfamily fn} keyword, followed by the function\textquotesingle{}s name, a list of parameters enclosed in parentheses, and the body of the function in curly braces. This example computes the square of a number\+: \begin{DoxyVerb}fn sqr(x) {
  return x*x    
}
\end{DoxyVerb}


\#\+Return \mbox{[}tag\mbox{]}\+: \# return

The {\ttfamily return} keyword is used to exit from a function, optionally passing a given value back to the caller. {\ttfamily return} can be used anywhere within a function. The below example calculates the {\ttfamily n} th Fibonacci number, \begin{DoxyVerb}fn fib(n) {
  if (n<2) return n 
  return fib(n-1) + fib(n-2) 
}
\end{DoxyVerb}


by returning early if {\ttfamily n$<$2}, otherwise returning the result by recursively calling itself.

\#\+Variables \mbox{[}tag\mbox{]}\+: \# var

Variables are defined using the {\ttfamily var} keyword followed by the variable name\+: \begin{DoxyVerb}var a
\end{DoxyVerb}


Optionally, an initial assignment may be given\+: \begin{DoxyVerb}var a = 1
\end{DoxyVerb}


Variables defined in a block of code are visible only within that block, so \begin{DoxyVerb}var greeting = "Hello" 
{
    var greeting = "Goodbye" 
    print greeting 
}
print greeting 
\end{DoxyVerb}


will print

{\itshape Goodbye} {\itshape Hello}

Multiple variables can be defined at once by separating them with commas \begin{DoxyVerb}var a, b=2, c[2]=[1,2]
\end{DoxyVerb}


where each can have its own initializer (or not).

\#\+Classes \mbox{[}tag\mbox{]}\+: \# class

Classes are defined using the {\ttfamily class} keyword followed by the name of the class. The definition includes methods that the class responds to. The special {\ttfamily init} method is called whenever an object is created. \begin{DoxyVerb}class Cake {
    init(type) {
        self.type = type
    }

    eat() {
        print "A delicious "+type+" cake"
    }
}
\end{DoxyVerb}


Objects are created by calling the class as if it was a function\+: \begin{DoxyVerb}var c = Cake("carrot")
\end{DoxyVerb}


Methods are called using the . operator\+: \begin{DoxyVerb}c.eat()
\end{DoxyVerb}


\#\+If \mbox{[}tag\mbox{]}\+: \# if

If allows you to selectively execute a section of code depending on whether a condition is met. The simplest version looks like this\+: \begin{DoxyVerb}if (x<1) print x
\end{DoxyVerb}


where the body of the loop, {\ttfamily print x}, is only executed if x is less than 1. The body can be a code block to accomodate longer sections of code\+: \begin{DoxyVerb}if (x<1) {
    ... // do something
}
\end{DoxyVerb}


If you want to choose between two alternatives, use {\ttfamily else}\+: \begin{DoxyVerb}if (a==b) {
    // do something
} else {
    // this code is executed only if the condition is false 
}
\end{DoxyVerb}


You can even chain multiple tests together like this\+: \begin{DoxyVerb}if (a==b) { 
    // option 1 
} else if (a==c) {
    // option 2 
} else {
    // something else 
}
\end{DoxyVerb}


\#\+While \mbox{[}tag\mbox{]}\+: \# while

While loops

\#\+For \mbox{[}tag\mbox{]}\+: \# for

For loops allow you to repeatedly execute a section of code. They come in two versions\+: the simpler version looks like this, \begin{DoxyVerb}for (i in 1..5) print i
\end{DoxyVerb}


which prints the numbers 1 to 5 in turn. The variable {\ttfamily i} is the {\itshape loop variable}, which takes on a different value each iteration. {\ttfamily 1..5} is a range, which denotes a sequence of numbers. The {\itshape body} of the loop, {\ttfamily print i}, is the code to be repeatedly executed. ~\newline


If you want your loop variable to count in increments other than 1, you can specify a stepsize in the range\+: \begin{DoxyVerb}for (i in 1..5:2) print i
               ^step 
\end{DoxyVerb}


Ranges need not be integer\+: \begin{DoxyVerb}for (i in 0.1..0.5:0.1) print i
\end{DoxyVerb}


You can also replace the range with other kinds of collection object to loop over their contents\+: \begin{DoxyVerb}var a = Matrix([1,2,3,4])
for (x in a) print x
\end{DoxyVerb}


Morpho also provides a second form of {\ttfamily for} loop similar to that in C\+: \begin{DoxyVerb}for (var i=0; i<5; i+=1) { print i }
     ^start   ^test ^inc.  ^body
\end{DoxyVerb}


which is executed as follows\+: start\+: the variable {\ttfamily i} is initially set to zero. test\+: before each iteration, the test is evaluated. If the test is {\ttfamily false}, the loop terminates. body\+: the body of the loop is executed. inc\+: the variable {\ttfamily i} is increased by 1.

This kind of loop is very flexible as you can include any code that you like in each of the sections.

\#\+Indexing \mbox{[}tag\mbox{]}\+: \# \mbox{[} \mbox{[}tag\mbox{]}\+: \# \mbox{]} \mbox{[}tag\mbox{]}\+: \# index \mbox{[}tag\mbox{]}\+: \# subscript

Indexing

\#\+Example

This is {\itshape very} important, this is {\itshape underlined} and {\ttfamily var} is code. 