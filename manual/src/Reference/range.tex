\hypertarget{range}{%
\section{Range}\label{range}}

Ranges represent a sequence of numerical values. There are two ways to
create them depending on whether the upper value is included or not:

\begin{lstlisting}
var a = 1..5  // inclusive version, i.e. [1,2,3,4,5]
var b = 1...5 // exclusive version, i.e. [1,2,3,4]
\end{lstlisting}

By default, the increment between values is 1, but you can use a
different value like this:

\begin{lstlisting}
var a = 1..5:0.5 // 1 - 5 with an increment of 0.5.
\end{lstlisting}

You can also create Range objects using the appropriate constructor
function:

\begin{lstlisting}
var a = Range(1,5,0.5)
\end{lstlisting}

Ranges are particularly useful in writing loops:

\begin{lstlisting}
for (i in 1..5) print i
\end{lstlisting}

They can easily be converted to a list of values:

\begin{lstlisting}
var c = List(1..5)
\end{lstlisting}

To find the number of elements in a Range, use the \texttt{count} method

\begin{lstlisting}
print (1..5).count()
\end{lstlisting}
