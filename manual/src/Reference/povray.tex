\hypertarget{povray}{%
\section{POVRay}\label{povray}}

The \texttt{povray} module provides integration with POVRay, a popular
open source ray-tracing package for high quality graphical rendering. To
use the module, first import it:

\begin{lstlisting}
import povray
\end{lstlisting}

To raytrace a graphic, begin by creating a \texttt{POVRaytracer} object:

\begin{lstlisting}
var pov = POVRaytracer(graphic)
\end{lstlisting}

Create a .pov file that can be run with POVRay:

\begin{lstlisting}
pov.write("out.pov")
\end{lstlisting}

Create, render and display a scene using POVRay:

\begin{lstlisting}
pov.render("out.pov")
\end{lstlisting}

This also creates the .png file for the scene.

The \texttt{POVRaytracer} constructor supports an optional
\texttt{camera} argument:

\begin{itemize}

\item
  \texttt{camera} - a \texttt{Camera} object (see below / help)
  containing the settings for the povray camera.
\end{itemize}

The \texttt{Camera} object can be initialized as follows:

\begin{lstlisting}
var camera = Camera()
\end{lstlisting}

This object contains the default settings of the camera, which can be
changed using the following optional arguments, or by just setting the
attributes after instantiation:

\begin{itemize}

\item
  \texttt{antialias} - whether to antialias the output or not
  (\texttt{true} by default)
\item
  \texttt{width} - image width (\texttt{2048} by default)
\item
  \texttt{height} - image height (\texttt{1536} by default)
\item
  \texttt{viewangle} - camera angle (higher means wider view)
  (\texttt{24} by default)
\item
  \texttt{viewpoint} - position of camera (\texttt{Matrix({[}0,0,-5{]})}
  by default)
\item
  \texttt{look\_at} - coordinate to look at
  (\texttt{Matrix({[}0,0,0{]})} by defualt)
\item
  \texttt{sky} - orientation pointing to the sky
  (\texttt{Matrix({[}0,1,0{]})} by default)
\end{itemize}

The default settings generate a reasonable centered view of the x-y
plane.

These attributes can also be set directly for the \texttt{POVRaytracer}
object:

\begin{lstlisting}
pov.look_at = Matrix([0,0,1])
\end{lstlisting}

The \texttt{render} method supports two optional boolean arguments:

\begin{itemize}

\item
  \texttt{quiet} - whether to suppress the parser and render statistics
  from \texttt{povray} or not (\texttt{false} by default)
\item
  \texttt{display} - whether to turn on the graphic display while
  rendering or not (\texttt{true} by default)
\end{itemize}

\hypertarget{camera}{%
\section{Camera}\label{camera}}

The \texttt{Camera} object can be initialized as follows:

\begin{lstlisting}
var camera = Camera()
\end{lstlisting}

This object contains the default settings of the camera, which can be
changed using the following optional arguments, or by just setting the
attributes after instantiation:

\begin{itemize}
\item
  \texttt{antialias} - whether to antialias the output or not
  (\texttt{true} by default)
\item
  \texttt{width} - image width (\texttt{2048} by default)
\item
  \texttt{height} - image height (\texttt{1536} by default)
\item
  \texttt{viewangle} - camera angle (higher means wider view)
  (\texttt{24} by default)
\item
  \texttt{viewpoint} - position of camera (\texttt{Matrix({[}0,0,-5{]})}
  by default)
\item
  \texttt{look\_at} - coordinate to look at
  (\texttt{Matrix({[}0,0,0{]})} by defualt)
\item
  \texttt{sky} - orientation pointing to the sky
  (\texttt{Matrix({[}0,1,0{]})} by default)

  camera.sky = Matrix({[}0,0,1{]})
\end{itemize}

The default settings generate a reasonable centered view of the x-y
plane.
