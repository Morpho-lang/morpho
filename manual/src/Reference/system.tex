\hypertarget{system}{%
\section{System}\label{system}}

The \texttt{System} class provides information and access to some
features of the runtime environment.

\hypertarget{platform}{%
\subsection{Platform}\label{platform}}

Detect which platform morpho was compiled for:

\begin{lstlisting}
print System.platform() 
\end{lstlisting}

which returns \texttt{"macos"}, \texttt{"linux"}, \texttt{"unix"} or
\texttt{"windows"}.

\hypertarget{version}{%
\subsection{Version}\label{version}}

Find the current version of morpho:

\begin{lstlisting}
print System.version() 
\end{lstlisting}

\hypertarget{clock}{%
\subsection{Clock}\label{clock}}

Returns the system time in seconds, with at least millisecond
granularity. Primarily intended for timing:

\begin{lstlisting}
var start=System.clock() 
// Do something 
print System.clock()-start 
\end{lstlisting}

Note that \texttt{System.clock} measures the actual physical time
elapsed, not the time spent in a process.

\hypertarget{sleep}{%
\subsection{Sleep}\label{sleep}}

Pauses the program for a specified number of seconds:

\begin{lstlisting}
System.sleep(0.5) // Sleep for half a second
\end{lstlisting}

\hypertarget{readline}{%
\subsection{Readline}\label{readline}}

Reads a line of input from the console:

\begin{lstlisting}
var in = System.readline() 
\end{lstlisting}

\hypertarget{arguments}{%
\subsection{Arguments}\label{arguments}}

Returns a \texttt{List} of arguments passed to the current morpho on the
command line.

\begin{lstlisting}
var args = System.arguments() 
for (e in args) print e 
\end{lstlisting}

Run a morpho program with arguments:

\begin{lstlisting}
morpho5 program.morpho hello world
\end{lstlisting}

Note that, in line with UNIX conventions, command line arguments before
the program file name are passed to the \texttt{morpho5} runtime; those
after are passed to the morpho program via \texttt{System.arguments}.

\hypertarget{exit}{%
\subsection{Exit}\label{exit}}

Stop execution of a program:

\begin{lstlisting}
System.exit() 
\end{lstlisting}
