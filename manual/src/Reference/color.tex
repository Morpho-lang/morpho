\hypertarget{color}{%
\section{Color}\label{color}}

The \texttt{color} module provides support for working with color.
Colors are represented in morpho by \texttt{Color} objects. The module
predefines some colors including \texttt{Red}, \texttt{Green},
\texttt{Blue}, \texttt{Black}, \texttt{White}.

To use the module, use import as usual:

\begin{lstlisting}
import color
\end{lstlisting}

Create a Color object from an RGB pair:

\begin{lstlisting}
var col = Color(0.5,0.5,0.5) // A 50% gray
\end{lstlisting}

\hypertarget{colormap}{%
\section{Colormap}\label{colormap}}

The \texttt{color} module provides \texttt{ColorMap}s which are
subclasses of \texttt{Color} that map a single parameter in the range
0..1 onto a continuum of colors. These include \texttt{GradientMap},
\texttt{GrayMap} and \texttt{HueMap}. \texttt{Color}s and
\texttt{Colormap}s have the same interface.

Get the red, green or blue components of a color or colormap:

\begin{lstlisting}
var col = HueMap()
print col.red(0.5) // argument can be in range 0..1
\end{lstlisting}

Get all three components as a list:

\begin{lstlisting}
col.rgb(0)
\end{lstlisting}

Create a grayscale:

\begin{lstlisting}
var c = Gray(0.2) // 20% gray
\end{lstlisting}

\hypertarget{rgb}{%
\section{RGB}\label{rgb}}

Gets the rgb components of a \texttt{Color} or \texttt{ColorMap} object
as a list. Takes a single argument in the range 0..1, although the
result will only depend on this argument if the object is a
\texttt{ColorMap}.

\begin{lstlisting}
var col = Color(0.1,0.5,0.7)
print col.rgb(0)
\end{lstlisting}
