\hypertarget{plot}{%
\section{Plot}\label{plot}}

The \texttt{plot} module provides visualization capabilities for Meshes,
Selections and Fields. These functions produce Graphics objects that can
be displayed with \texttt{Show}.

To use the module, first import it:

\begin{lstlisting}
import plot
\end{lstlisting}

\hypertarget{plotmesh}{%
\subsection{Plotmesh}\label{plotmesh}}

Visualizes a \texttt{Mesh} object:

\begin{lstlisting}
var g = plotmesh(mesh)
\end{lstlisting}

Plotmesh accepts a number of optional arguments to control what is
displayed:

\begin{itemize}

\item
  \texttt{selection} - Only elements in a provided \texttt{Selection}
  are drawn.
\item
  \texttt{grade} - Only draw the specified grade. This can also be a
  list of multiple grades to draw.
\item
  \texttt{color} - Draw the mesh in a provided \texttt{Color}.
\item
  \texttt{filter} and \texttt{transmit} - Used by the \texttt{povray}
  module to indicate transparency.
\end{itemize}

\hypertarget{plotmeshlabels}{%
\subsection{Plotmeshlabels}\label{plotmeshlabels}}

Draws the ids for elements in a \texttt{Mesh}:

\begin{lstlisting}
var g = plotmeshlabels(mesh) 
\end{lstlisting}

Plotmeshlabels accepts a number of optional arguments to control the
output:

\begin{itemize}

\item
  \texttt{grade} - Only draw the specified grade. This can also be a
  list of multiple grades to draw.
\item
  \texttt{selection} - Only labels in a provided \texttt{Selection} are
  drawn.
\item
  \texttt{offset} - Local offset vector for labels. Can be a
  \texttt{List}, a \texttt{Matrix} or a function.
\item
  \texttt{dirn} - Text direction for labels. Can be a \texttt{List}, a
  \texttt{Matrix} or a function.
\item
  \texttt{vertical} - Text vertical direction. Can be a \texttt{List}, a
  \texttt{Matrix} or a function.
\item
  \texttt{color} - Label color. Can be a \texttt{Color} object or a
  \texttt{Dictionary} of colors for each grade.
\item
  \texttt{fontsize} - Font size to use.
\end{itemize}

\hypertarget{plotselection}{%
\subsection{Plotselection}\label{plotselection}}

Visualizes a \texttt{Selection} object:

\begin{lstlisting}
var g = plotselection(mesh, sel)
\end{lstlisting}

Plotselection accepts a number of optional arguments to control what is
displayed:

\begin{itemize}

\item
  \texttt{grade} - Only draw the specified grade. This can also be a
  list of multiple grades to draw.
\item
  \texttt{filter} and \texttt{transmit} - Used by the \texttt{povray}
  module to indicate transparency.
\end{itemize}

\hypertarget{plotfield}{%
\subsection{Plotfield}\label{plotfield}}

Visualizes a scalar \texttt{Field} object:

\begin{lstlisting}
var g = plotfield(field)
\end{lstlisting}

Plotfield accepts a number of optional arguments to control what is
displayed:

\begin{itemize}

\item
  \texttt{grade} - Draw the specified grade.
\item
  \texttt{colormap} - A \texttt{Colormap} object to use. The field is
  automatically scaled.
\item
  \texttt{scalebar} - A \texttt{Scalebar} object to use.
\item
  \texttt{style} - Plot style. See below.
\item
  \texttt{filter} and \texttt{transmit} - Used by the \texttt{povray}
  module to indicate transparency.
\item
  \texttt{cmin} and \texttt{cmax} - Can be used to define the data range
  covered. Values beyond these limits will be colored by the lower/upper
  bound of the colormap accordingly.
\end{itemize}

Supported plot styles:

\begin{itemize}

\item
  \texttt{default} - Color \texttt{Mesh} elements by the corresponding
  value of the \texttt{Field}.
\item
  \texttt{interpolate} - Interpolate \texttt{Field} quantities onto
  higher elements.
\end{itemize}

\hypertarget{scalebar}{%
\subsection{ScaleBar}\label{scalebar}}

Represents a scalebar for a plot:

\begin{lstlisting}
Show(plotfield(field, scalebar=ScaleBar(posn=[1.2,0,0])))
\end{lstlisting}

\texttt{ScaleBar}s can be created with many adjustable parameters:

\begin{itemize}

\item
  \texttt{nticks} - Maximum number of ticks to show.\\
\item
  \texttt{posn} - Position to draw the \texttt{ScaleBar}.
\item
  \texttt{length} - Length of \texttt{ScaleBar} to draw.
\item
  \texttt{dirn} - Direction to draw the \texttt{ScaleBar} in.
\item
  \texttt{tickdirn} - Direction to draw the ticks in.
\item
  \texttt{colormap} - \texttt{ColorMap} to use.
\item
  \texttt{textdirn} - Direction to draw labels in.
\item
  \texttt{textvertical} - Label vertical direction.
\item
  \texttt{fontsize} - Fontsize for labels
\item
  \texttt{textcolor} - Color for labels
\end{itemize}

You can draw the \texttt{ScaleBar} directly by calling the \texttt{draw}
method:

\begin{lstlisting}
sb.draw(min, max)
\end{lstlisting}

where \texttt{min} and \texttt{max} are the minimum and maximum values
to display on the scalebar.
