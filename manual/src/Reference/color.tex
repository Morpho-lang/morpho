\hypertarget{color}{%
\section{Color}\label{color}}

The \texttt{color} module provides support for working with color.
Colors are represented in morpho by \texttt{Color} objects. The module
predefines some colors including \texttt{Red}, \texttt{Green},
\texttt{Blue}, \texttt{Black}, \texttt{White}.

To use the module, use import as usual:

\begin{lstlisting}
import color
\end{lstlisting}

Create a Color object from an RGB pair:

\begin{lstlisting}
var col = Color(0.5,0.5,0.5) // A 50% gray
\end{lstlisting}

The \texttt{color} module also provides \texttt{ColorMap}s, which are
give a sequence of colors as a function of a parameter; these are useful
for plotting the values of a \texttt{Field} for example.

\hypertarget{rgb}{%
\subsection{RGB}\label{rgb}}

Gets the rgb components of a \texttt{Color} or \texttt{ColorMap} object
as a list. Takes a single argument in the range 0 to 1, although the
result will only depend on this argument if the object is a
\texttt{ColorMap}.

\begin{lstlisting}
var col = Color(0.1,0.5,0.7)
print col.rgb(0)
\end{lstlisting}

\hypertarget{red}{%
\subsection{Red}\label{red}}

Built in \texttt{Color} object for use with the \texttt{graphics} and
\texttt{plot} modules.

\hypertarget{green}{%
\subsection{Green}\label{green}}

Built in \texttt{Color} object for use with the \texttt{graphics} and
\texttt{plot} modules.

\hypertarget{blue}{%
\subsection{Blue}\label{blue}}

Built in \texttt{Color} object for use with the \texttt{graphics} and
\texttt{plot} modules.

\hypertarget{white}{%
\subsection{White}\label{white}}

Built in \texttt{Color} object for use with the \texttt{graphics} and
\texttt{plot} modules.

\hypertarget{black}{%
\subsection{Black}\label{black}}

Built in \texttt{Color} object for use with the \texttt{graphics} and
\texttt{plot} modules.

\hypertarget{cyan}{%
\subsection{Cyan}\label{cyan}}

Built in \texttt{Color} object for use with the \texttt{graphics} and
\texttt{plot} modules.

\hypertarget{magenta}{%
\subsection{Magenta}\label{magenta}}

Built in \texttt{Color} object for use with the \texttt{graphics} and
\texttt{plot} modules.

\hypertarget{yellow}{%
\subsection{Yellow}\label{yellow}}

Built in \texttt{Color} object for use with the \texttt{graphics} and
\texttt{plot} modules.

\hypertarget{brown}{%
\subsection{Brown}\label{brown}}

Built in \texttt{Color} object for use with the \texttt{graphics} and
\texttt{plot} modules.

\hypertarget{orange}{%
\subsection{Orange}\label{orange}}

Built in \texttt{Color} object for use with the \texttt{graphics} and
\texttt{plot} modules.

\hypertarget{pink}{%
\subsection{Pink}\label{pink}}

Built in \texttt{Color} object for use with the \texttt{graphics} and
\texttt{plot} modules.

\hypertarget{purple}{%
\subsection{Purple}\label{purple}}

Built in \texttt{Color} object for use with the \texttt{graphics} and
\texttt{plot} modules.

\hypertarget{colormap}{%
\subsection{Colormap}\label{colormap}}

The \texttt{color} module provides \texttt{ColorMap}s which are
subclasses of \texttt{Color} that map a single parameter in the range 0
to 1 onto a continuum of colors. \texttt{Color}s and \texttt{Colormap}s
have the same interface.

Get the red, green or blue components of a color or colormap:

\begin{lstlisting}
var col = HueMap()
print col.red(0.5) // argument can be in range 0 to 1
\end{lstlisting}

Get all three components as a list:

\begin{lstlisting}
col.rgb(0)
\end{lstlisting}

Create a grayscale:

\begin{lstlisting}
var c = Gray(0.2) // 20% gray
\end{lstlisting}

Available ColorMaps: \texttt{GradientMap}, \texttt{GrayMap},
\texttt{HueMap}, \texttt{ViridisMap}, \texttt{MagmaMap},
\texttt{InfernoMap} and \texttt{PlasmaMap}.

\hypertarget{gradientmap}{%
\subsection{GradientMap}\label{gradientmap}}

\texttt{GradientMap} is a \texttt{Colormap} that displays a
white-green-purple sequence.

\hypertarget{graymap}{%
\subsection{GrayMap}\label{graymap}}

\texttt{GrayMap} is a \texttt{Colormap} that displays grayscales.

\hypertarget{huemap}{%
\subsection{HueMap}\label{huemap}}

\texttt{HueMap} is a \texttt{Colormap} that displays vivid colors. It is
periodic on the interval 0 to 1.

\hypertarget{viridismap}{%
\subsection{ViridisMap}\label{viridismap}}

\texttt{ViridisMap} is a \texttt{Colormap} that displays a
purple-green-yellow sequence. It is perceptually uniform and intended to
be improve the accessibility of visualizations for viewers with color
vision deficiency.

\hypertarget{magmamap}{%
\subsection{MagmaMap}\label{magmamap}}

\texttt{MagmaMap} is a \texttt{Colormap} that displays a
black-red-yellow sequence. It is perceptually uniform and intended to be
improve the accessibility of visualizations for viewers with color
vision deficiency.

\hypertarget{infernomap}{%
\subsection{InfernoMap}\label{infernomap}}

\texttt{InfernoMap} is a \texttt{Colormap} that displays a
black-red-yellow sequence. It is perceptually uniform and intended to be
improve the accessibility of visualizations for viewers with color
vision deficiency.

\hypertarget{plasmamap}{%
\subsection{PlasmaMap}\label{plasmamap}}

\texttt{InfernoMap} is a \texttt{Colormap} that displays a
blue-red-yellow sequence. It is perceptually uniform and intended to be
improve the accessibility of visualizations for viewers with color
vision deficiency.
