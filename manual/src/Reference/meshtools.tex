\hypertarget{meshtools}{%
\section{Meshtools}\label{meshtools}}

The Meshtools package contains a number of functions and classes to
assist with creating and manipulating meshes.

\hypertarget{areamesh}{%
\subsection{AreaMesh}\label{areamesh}}

This function creates a mesh composed of triangles from a parametric
function. To use it:

\begin{lstlisting}
var m = AreaMesh(function, range1, range2, closed=boolean)
\end{lstlisting}

where

\begin{itemize}

\item
  \texttt{function} is a parametric function that has one parameter. It
  should return a list of coordinates or a column matrix corresponding
  to this parameter.
\item
  \texttt{range1} is the Range to use for the first parameter of the
  parametric function.
\item
  \texttt{range2} is the Range to use for the second parameter of the
  parametric function.
\item
  \texttt{closed} is an optional parameter indicating whether to create
  a closed loop or not. You can supply a list where each element
  indicates whether the relevant parameter is closed or not.
\end{itemize}

To use \texttt{AreaMesh}, import the \texttt{meshtools} module:

\begin{lstlisting}
import meshtools
\end{lstlisting}

Create a square:

\begin{lstlisting}
var m = AreaMesh(fn (u,v) [u, v, 0], 0..1:0.1, 0..1:0.1)
\end{lstlisting}

Create a tube:

\begin{lstlisting}
var m = AreaMesh(fn (u, v) [v, cos(u), sin(u)], -Pi...Pi:Pi/4,
                 -1..1:0.1, closed=[true, false])
\end{lstlisting}

Create a torus:

\begin{lstlisting}
var c=0.5, a=0.2
var m = AreaMesh(fn (u, v) [(c + a*cos(v))*cos(u),
                            (c + a*cos(v))*sin(u),
                            a*sin(v)], 0...2*Pi:Pi/16, 0...2*Pi:Pi/8, closed=true)
\end{lstlisting}

\hypertarget{linemesh}{%
\subsection{LineMesh}\label{linemesh}}

This function creates a mesh composed of line elements from a parametric
function. To use it:

\begin{lstlisting}
var m = LineMesh(function, range, closed=boolean)
\end{lstlisting}

where

\begin{itemize}

\item
  \texttt{function} is a parametric function that has one parameter. It
  should return a list of coordinates or a column matrix corresponding
  to this parameter.
\item
  \texttt{range} is the Range to use for the parametric function.
\item
  \texttt{closed} is an optional parameter indicating whether to create
  a closed loop or not.
\end{itemize}

To use \texttt{LineMesh}, import the \texttt{meshtools} module:

\begin{lstlisting}
import meshtools
\end{lstlisting}

Create a circle:

\begin{lstlisting}
import constants
var m = LineMesh(fn (t) [sin(t), cos(t), 0], 0...2*Pi:2*Pi/50, closed=true)
\end{lstlisting}

\hypertarget{polyhedronmesh}{%
\subsection{PolyhedronMesh}\label{polyhedronmesh}}

This function creates a mesh corresponding to a polyhedron.

\begin{lstlisting}
var m = PolyhedronMesh(vertices, faces)
\end{lstlisting}

where \texttt{vertices} is a list of vertices and \texttt{faces} is a
list of faces specified as a list of vertex indices.

To use \texttt{PolyhedronMesh}, import the \texttt{meshtools} module:

\begin{lstlisting}
import meshtools
\end{lstlisting}

Create a cube:

\begin{lstlisting}
var m = PolyhedronMesh([ [-0.5, -0.5, -0.5], [ 0.5, -0.5, -0.5],
                         [-0.5,  0.5, -0.5], [ 0.5,  0.5, -0.5],
                         [-0.5, -0.5,  0.5], [ 0.5, -0.5,  0.5],
                         [-0.5,  0.5,  0.5], [ 0.5,  0.5,  0.5]],
                       [ [0,1,3,2], [4,5,7,6], [0,1,5,4], 
                         [3,2,6,7], [0,2,6,4], [1,3,7,5] ])
\end{lstlisting}

\emph{Note} that the vertices in each face list must be specified
strictly in cyclic order.

\hypertarget{delaunaymesh}{%
\subsection{DelaunayMesh}\label{delaunaymesh}}

The \texttt{DelaunayMesh} constructor function creates a \texttt{Mesh}
object directly from a point cloud using the Delaunay triangulator.

\begin{lstlisting}
var pts = []
for (i in 0...100) pts.append(Matrix([random(), random()]))
var m=DelaunayMesh(pts)
Show(plotmesh(m))
\end{lstlisting}

You can control the output dimension of the mesh (e.g.~to create a 2D
mesh embedded in 3D space) using the optional \texttt{outputdim}
property.

\begin{lstlisting}
var m = DelaunayMesh(pts, outputdim=3)
\end{lstlisting}

\hypertarget{equiangulate}{%
\subsection{Equiangulate}\label{equiangulate}}

Attempts to equiangulate a mesh, exchanging elements to improve their
regularity.

\begin{lstlisting}
equiangulate(mesh)
\end{lstlisting}

\emph{Note} this function modifies the mesh in place; it does not create
a new mesh.

\hypertarget{changemeshdimension}{%
\subsection{ChangeMeshDimension}\label{changemeshdimension}}

Changes the dimension in which a mesh is embedded. For example, you may
have created a mesh in 2D that you now wish to use in 3D.

To use:

\begin{lstlisting}
var new = ChangeMeshDimension(mesh, dim)
\end{lstlisting}

where \texttt{mesh} is the mesh you wish to change, and \texttt{dim} is
the new embedding dimension.

\hypertarget{meshbuilder}{%
\subsection{MeshBuilder}\label{meshbuilder}}

The \texttt{MeshBuilder} class simplifies user creation of meshes. To
use this class, begin by creating a \texttt{MeshBuilder} object:

\begin{lstlisting}
var build = MeshBuilder()
\end{lstlisting}

You can then add vertices, edges, etc. one by one using
\texttt{addvertex}, \texttt{addedge}, \texttt{addface} and
\texttt{addelement}. Each of these returns an element id:

\begin{lstlisting}
var id1=build.addvertex(Matrix([0,0,0]))
var id2=build.addvertex(Matrix([1,1,1]))
build.addedge([id1, id2])
\end{lstlisting}

Once the mesh is ready, call the \texttt{build} method to construct the
\texttt{Mesh}:

\begin{lstlisting}
var m = build.build()
\end{lstlisting}

You can specify the dimension of the \texttt{Mesh} explicitly when
initializing the \texttt{MeshBuilder}:

\begin{lstlisting}
var mb = MeshBuilder(dimension=2)
\end{lstlisting}

or implicitly when adding the first vertex:

\begin{lstlisting}
var mb = MeshBuilder() 
mb.addvertex([0,1]) // A 2D mesh
\end{lstlisting}

\hypertarget{mshblddimincnstnt}{%
\subsection{MshBldDimIncnstnt}\label{mshblddimincnstnt}}

This error is produced if you try to add a vertex that is inconsistent
with the mesh dimension, e.g.

\begin{lstlisting}
var mb = MeshBuilder(dimension=2) 
mb.addvertex([1,0,0]) // Throws an error! 
\end{lstlisting}

To fix this ensure all vertices have the correct dimension.

\hypertarget{mshblddimunknwn}{%
\subsection{MshBldDimUnknwn}\label{mshblddimunknwn}}

This error is produced if you try to add an element to a
\texttt{MeshBuilder} object but haven't yet specified the dimension (at
initialization) or by adding a vertex.

\begin{lstlisting}
var mb = MeshBuilder() 
mb.addedge([0,1]) // No vertices have been added 
\end{lstlisting}

To fix this add the vertices first.

\hypertarget{meshrefiner}{%
\subsection{MeshRefiner}\label{meshrefiner}}

The \texttt{MeshRefiner} class is used to refine meshes, and to correct
associated data structures that depend on the mesh.

To prepare for refining, first create a \texttt{MeshRefiner} object
either with a \texttt{Mesh},

\begin{lstlisting}
var mr = MeshRefiner(mesh)
\end{lstlisting}

or with a list of objects that can include a \texttt{Mesh} as well as
\texttt{Field}s and \texttt{Selection}s.

\begin{lstlisting}
var mr = MeshRefiner([mesh, field, selection ... ])
\end{lstlisting}

To perform the refinement, call the \texttt{refine} method. You can
refine all elements,

\begin{lstlisting}
var dict = mr.refine()
\end{lstlisting}

or refine selected elements using a \texttt{Selection},

\begin{lstlisting}
var dict = mr.refine(selection=select)
\end{lstlisting}

The \texttt{refine} method returns a \texttt{Dictionary} that maps old
objects to new, refined objects. Use this to update your data
structures.

\begin{lstlisting}
var newmesh = dict[oldmesh]
\end{lstlisting}

\hypertarget{meshpruner}{%
\subsection{MeshPruner}\label{meshpruner}}

The \texttt{MeshPruner} class is used to prune excessive detail from
meshes (a process that's sometimes referred to as coarsening), and to
correct associated data structures that depend on the mesh.

First create a \texttt{MeshPruner} object either with a \texttt{Mesh},

\begin{lstlisting}
var mp = MeshPruner(mesh)
\end{lstlisting}

or with a list of objects that can include a \texttt{Mesh} as well as
\texttt{Field}s and \texttt{Selection}s.

\begin{lstlisting}
var mp = MeshPruner([mesh, field, selection ... ])
\end{lstlisting}

To perform the coarsening, call the \texttt{prune} method with a
\texttt{Selection},

\begin{lstlisting}
var dict = mp.prune(select)
\end{lstlisting}

The \texttt{prune} method returns a \texttt{Dictionary} that maps old
objects to new, refined objects. Use this to update your data
structures.

\begin{lstlisting}
var newmesh = dict[oldmesh]
\end{lstlisting}

\hypertarget{meshmerge}{%
\subsection{MeshMerge}\label{meshmerge}}

The \texttt{MeshMerge} class is used to combine meshes into a single
mesh, removing any duplicate elements.

To use, create a \texttt{MeshMerge} object with a list of meshes to
merge,

\begin{lstlisting}
var mrg = MeshMerge([m1, m2, m3, ... ])
\end{lstlisting}

and then call the \texttt{merge} method to return a combined mesh:

\begin{lstlisting}
var newmesh = mrg.merge()
\end{lstlisting}
